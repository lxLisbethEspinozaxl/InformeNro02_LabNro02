\section{Actividad No 02 – Reconociendo la estructura} 

\begin{enumerate}[1.]
	\item Crear la tabla Departamentos utilizando la siguiente estructura:
	\\
		\begin{center}
		\includegraphics[width=10cm]{./Imagenes/enun1} 
		\end{center}
	
		\begin{center}
		\includegraphics[width=10cm]{./Imagenes/eje1} 
		\end{center}
	
	\item Poblar la tabla Departamentos con los datos de la tabla Departments.
	\\
	\begin{center}
	\includegraphics[width=10cm]{./Imagenes/eje2} 
	\end{center}
	
	\item Crear la tabla Empleados utilizando la siguiente estructura.
	\\
	\begin{center}
	\includegraphics[width=10cm]{./Imagenes/enun3} 
	\end{center}
	\begin{center}
	\includegraphics[width=10cm]{./Imagenes/eje3} 
	\end{center}
	
	\item Crear la tabla Empleados2 basada en la estructura de la tabla Employees. Incluir solo las columnas EMPLOYEE\_ID, FIRST\_NAME, LAST\_NAME, SALARY y DEPARMENT\_ID 		respectivamente.
	\\
	\begin{center}
	\includegraphics[width=10cm]{./Imagenes/eje4} 
	\end{center}
	
	\item Modificar el estado de la tabla Empleados2 a SOLO LECTURA.
	\\
	
	\item Tratar de adicionar el siguiente registro a la tabla Empleados2.
	\\
	\begin{center}
	\includegraphics[width=10cm]{./Imagenes/eje6} 
	\end{center}

	
	\item Revertir el estado de la tabla LECTURA / ESCRITURA. Tratar de insertar nuevamente la información del punto 4.6.
	
	\item Eliminar la tabla Empleados2.
	\\
	\begin{center}
	\includegraphics[width=10cm]{./Imagenes/eje8} 
	\end{center}
	
\end{enumerate}